\documentclass{beamer}
\usetheme{Madrid}
\AtBeginSection[]{
  \begin{frame}
  \vfill
  \centering
  \begin{beamercolorbox}[sep=8pt,center,shadow=true,rounded=true]{title}
    \usebeamerfont{title}\insertsectionhead\par%
  \end{beamercolorbox}
  \vfill
  \end{frame}
}

\usepackage[utf8]{vietnam}
\newcommand*{\doi}[1]{\href{http://dx.doi.org/#1}{doi: #1}}
\addtocontents{toc}{\setcounter{tocdepth}{2}} 
\usepackage{hyperref}
\usepackage{multirow}
\usecolortheme{default}
\setbeamertemplate{caption}[numbered]
\usepackage{subcaption}
\captionsetup{compatibility=false}
\usepackage{graphicx}
\usepackage{amsmath}

\title[Fall Detection System]
{Nghiên cứu về việc Tích hợp Trí tuệ Nhân tạo (AI) và Internet vạn vật (IoT) trong Hệ thống Phát hiện Té ngã theo Thời gian Thực}

\subtitle{Fall Detection Systems for Internet of Medical Things Based on Wearable Sensors: A Review (Zhiyuan Jiang, Mohammed A. A. Al-qaness*, Dalal AL-Alimi, Ahmed A. Ewess, Mohamed Abd Elaziz, Abdelghani Dahou, Ahmed M. Helmi)}


\author[Nguyen Dinh Anh] % (optional, for multiple authors)
{Nguyễn Đình Ánh}

\institute[HUTECH] % (optional)
{
  %
  Bộ môn Công nghệ phần mềm \\
  Khoa Công nghệ Thông tin\\
  Đại học Công nghệ TP. Hồ Chí Minh (HUTECH)
}

\date[HUTECH 2024] % (optional)
{\today}

\logo{\includegraphics[height=1cm]{logo_HUTECH.png}}

\definecolor{uoftblue}{RGB}{6,41,88}
\setbeamercolor{titlelike}{bg=uoftblue}
\setbeamerfont{title}{series=\bfseries}

\begin{document}

\frame{\titlepage}

\begin{frame}
\frametitle{Nội dung trình bày}
\tableofcontents
\end{frame}


\section{Giới Thiệu}
\begin{frame}{Giới Thiệu Hệ Thống Phát Hiện Ngã}
    \begin{itemize}
    \item \textbf{Vai trò của hệ thống phát hiện ngã}: Việc tích hợp hệ thống phát hiện ngã với công nghệ Internet of Things (IoT) đã tạo ra một bước tiến lớn trong lĩnh vực chăm sóc sức khỏe và an toàn cá nhân. Những hệ thống này giúp giám sát và hỗ trợ kịp thời, đặc biệt là đối với người cao tuổi và những người có nguy cơ ngã cao \footnote{M. N. Bhuiyan et al. (2021) \doi{10.1109/JIOT.2021.3062630}}.      
    \end{itemize}    
\end{frame}

\begin{frame}{Giới Thiệu Hệ Thống Phát Hiện Ngã}

    \begin{itemize}    
    \item \textbf{Kỹ thuật cảm biến đeo}: Các thiết bị cảm biến đeo được kết nối IoT có khả năng truyền tải dữ liệu theo thời gian thực và gửi thông báo tới người thân hoặc dịch vụ y tế khi phát hiện ngã, đảm bảo sự trợ giúp nhanh chóng \footnote{M. M. Islam et al. (2022) \doi{10.1109/JIOT.2022.3228795}}.
    \end{itemize}
    \begin{figure}
        \centering
        \includegraphics[width=.55\linewidth]{fig/IoT-fall-detection.png}
        \caption{Hệ thống phát hiện té ngã kết hợp IoT}
        \label{fig:lstm}
    \end{figure}
\end{frame}

\begin{frame}{Phương Pháp Phát Hiện Ngã}
    \begin{itemize}
    \item \textbf{Các phương pháp phát hiện ngã}: Các phương pháp phát hiện ngã dựa trên cảm biến đeo, phân loại thành ba nhóm chính: phương pháp dựa trên ngưỡng, phương pháp học máy truyền thống và phương pháp học sâu \footnote{M. M. Islam et al. (2022) \doi{10.1109/JIOT.2022.3228795}}.  
    \end{itemize}
    \begin{figure}
        \centering
        \includegraphics[width=.75\linewidth]{fig/fall-detection-method.png}
        \caption{Phương pháp phát hiện té ngã}
        \label{fig:lstm}
    \end{figure}
\end{frame}


\section{Tổng quan}
\begin{frame}{Hệ Thống Phát Hiện Ngã}
    \begin{itemize}
    \item \textbf{Định nghĩa ngã theo WHO}: Ngã là sự rơi đột ngột xuống mặt đất hoặc bề mặt khác mà không có tác động từ bên ngoài. Phát hiện ngã là một trong những biện pháp quan trọng nhằm bảo vệ sức khỏe và tăng khả năng sống sót của người cao tuổi. Các hệ thống này giúp phát hiện và can thiệp kịp thời khi có sự cố xảy ra \footnote{M. Abd Elaziz et al. (2024) \doi{10.1016/j.bspc.2024.106412}}.
    \end{itemize}
    
\end{frame}

\begin{frame}{Hệ Thống Phát Hiện Ngã}

    \begin{itemize}
    \item \textbf{Vai trò của IoT}: Khi tích hợp với IoT, hệ thống phát hiện ngã có thể kết nối với các thiết bị khác trong môi trường thông minh như nhà ở hoặc bệnh viện. Điều này cho phép hệ thống thu thập dữ liệu thời gian thực và gửi thông báo ngay lập tức tới người thân hoặc dịch vụ y tế, đảm bảo hỗ trợ nhanh chóng \footnote{A. Dahou et al. (2024) \doi{10.1109/JIOT.2023.3286378}}.
    \end{itemize}
    
\end{frame}

\begin{frame}{Hệ Thống Phát Hiện Ngã}
    \begin{itemize}
    \item \textbf{Các công nghệ cảm biến}: Các cảm biến phổ biến như gia tốc kế, áp suất và radar giúp phát hiện nhanh chóng các tình huống ngã. Những cảm biến này giám sát chuyển động và vị trí cơ thể, phát hiện các thay đổi đột ngột và gửi cảnh báo ngay khi phát hiện ngã, từ đó giúp giảm thiểu thiệt hại do ngã gây ra \footnote{S. Campanella et al. (2024) \doi{10.1109/JSEN.2024.3375603}}.    
    \end{itemize}    
\end{frame}

\begin{frame}{Hệ Thống Phát Hiện Ngã}

    \begin{itemize}
    \item \textbf{Phân loại hệ thống}: Hệ thống phát hiện ngã được chia thành ba loại chính: dựa trên cảm biến đeo, dựa trên camera, và dựa trên môi trường. Hệ thống cảm biến đeo được sử dụng rộng rãi nhờ tính linh hoạt và độ chính xác, trong khi các hệ thống dựa trên môi trường cung cấp giám sát liên tục mà không cần người dùng mang thiết bị \footnote{S. Wang et al. (2023) \doi{10.3390/s23146360}}.
    \begin{figure}
        \centering
        \includegraphics[width=.75\linewidth]{fig/methodologies-fall-detection.png}
        \caption{Các phương pháp triển khai}
        \label{fig:lstm}
    \end{figure}
    \end{itemize}
\end{frame}


\begin{frame}{}
  \centering \Huge
  \textbf{Thank You for Your Listening}
\end{frame}

\section{Tài liệu tham khảo}

\begin{frame}{Reference}
\scriptsize
    \begin{itemize}
        \item [1] M. N. Bhuiyan, M. M. Rahman, M. M. Billah, and D. Saha, “Internet of things (iot): A review of its enabling technologies in healthcare applications, standards protocols, security, and market opportunities,” IEEE Internet of Things Journal, vol. 8, no. 13, pp. 10 474–10 498, 2021.    
        \item [2] M. M. Islam, S. Nooruddin, F. Karray, and G. Muhammad, “Internet of things: Device capabilities, architectures, protocols, and smart applica-tions in healthcare domain,” IEEE Internet of Things Journal,vol. 10, no. 4, pp. 3611–3641, 2022.
        \item [3] S. Nooruddin, M. M. Islam, F. A. Sharna, H. Alhetari, and M. N. Kabir, “Sensor-based fall detection systems: a review,” Journal of Ambient Intelligence and Humanized Computing,pp. 1–17, 2022.
        \item [4] M. A. Al-qaness, A. Dahou, M. Abd Elaziz, and A. M. Helmi, “Human activity recognition and fall detection using convolutional neural network and transformer-based architecture,” Biomedical Signal Processing and Control,vol. 95, p. 106412, 2024.
        \item [5] A. Dahou, M. A. Al-Qaness, M. Abd Elaziz, and A. M. Helmi, “Mlcnnwav: Multi-level convolutional neural network with wavelet transformations for sensor-based human activity recognition,” IEEE Internet of Things Journal,vol. 11, no. 1, pp. 820–828, 2024.
        \item [6] S. Campanella, A. Alnasef, L. Falaschetti, A. Belli, P. Pierleoni, and L. Palma, “A novel embedded deep learning wearable sensor for fall detection,” IEEE Sensors Journal,2024.       
    \end{itemize}
\end{frame}


\end{document}